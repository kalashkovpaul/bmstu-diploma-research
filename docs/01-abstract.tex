\section*{РЕФЕРАТ}

Расчётно-пояснительная записка содержит \pageref{LastPage} с., \totalfigures\ рис., \totaltables\ табл., TODO ист.

ОБЪЕКТНАЯ МОДЕЛЬ ДОКУМЕНТА, ВИРТУАЛЬНАЯ ОБЪЕКТНАЯ МОДЕЛЬ ДОКУМЕНТА, ОБНОВЛЕНИЕ ГИПЕРТЕКСТОВОГО ДОКУМЕНТА, DOM, VDOM, АЛГОРИТМ СОГЛАСОВАНИЯ

Целью работы: анализ существующих методов обновления гипертекстовых документов, формализация алгоритма обновления гипертекстового документа Fiber, его анализ и сравнение его эффективности по сравнению с другими существующими алгоритмами.

%В данной работе проводится изучение принципов работы объектной модели документа (DOM), виртуальной объектной модели документа (VDOM), а также сравнение и анализ трудёмкостей алгоритмов обновления документа с использованием объектной модели документа и виртуальной объектной модели документа.

Результаты: TODO
% алгоритм обновления гипертекстового документа с использованием VDOM и алгоритма согласования имеет меньшую трудоёмкость при соблюдении эвриситки агоритма согласования, за счёт перерисовки только необходимых улов.
%В противном случае, когда эвристика алгоритма согласования не соблюдена, использование VDOM может иметь большую трудоёмкость, чем просто использование DOM.


\pagebreak