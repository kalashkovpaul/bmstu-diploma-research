\section*{РЕФЕРАТ}

Расчётно-пояснительная записка содержит \pageref{LastPage} с., \totalfigures\ рис., 36 ист.

ОБЪЕКТНАЯ МОДЕЛЬ ДОКУМЕНТА, ВИРТУАЛЬНАЯ ОБЪЕКТНАЯ МОДЕЛЬ ДОКУМЕНТА, ОБНОВЛЕНИЕ ГИПЕРТЕКСТОВОГО ДОКУМЕНТА, DOM, VDOM, АЛГОРИТМ СОГЛАСОВАНИЯ, FIBER

Целью работы: анализ существующих методов обновления гипертекстовых документов, формализация алгоритма обновления гипертекстового документа Fiber, его анализ и сравнение его эффективности по сравнению с другими существующими алгоритмами.

В данной работе проводится изучение принципов работы объектной модели документа (DOM), виртуальной объектной модели документа (VDOM), а также сравнение и анализ трудёмкостей алгоритмов обновления документа с использованием объектной модели документа и виртуальной объектной модели документа, а также алгоритма Fiber.

Результаты: из перечисленных алгоритмов именно алгоритм Fiber позволяет обеспечивать плавность обновления изображения на экране и при большом размере DOM-дерева, и при большом количестве операций обновления, в то время как алгоритм обновления с использованием алгоритма согласования выполняется от начала до конца и при большом количестве операций обновления может приводить к уменьшению количества кадров в секунду.

\pagebreak