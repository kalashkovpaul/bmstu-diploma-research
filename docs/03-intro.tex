\section*{ВВЕДЕНИЕ}
\addcontentsline{toc}{section}{ВВЕДЕНИЕ}

Работа с гипертекстовыми документами является неотъемлемой частью жизни каждого человека, пользующегося Всемирной сетью, и часто появляется потребность в просмотре различных гипертекстовых документов и в выполнении операций, приводящих к их изменению~\cite{hypertext-popular}.
Возникает вопрос: каким образом стоит производить операции обновления гипертекстовых документов?

Для взаимодействия с гипертекстовыми документами, входящими в сеть Интернет, существуют программы-браузеры.
Преимущественнная часть браузеров использует стандарт~\cite{dom-doc}, обеспечивающий использование объектной модели документа~\cite{dom}. 


\textbf{Целью данной работы} является формализация алгоритма обновления гипертекстового документа Fiber, использующего виртуальную объектную модель документа, а также анализ эффективности его использования в современных web-приложениях.
Для достижения поставленной цели необходимо выполнить следующие задачи:

\begin{enumerate}[label=\arabic*)]
	\item проанализировать существующие методы обновления гипертекстовых документов;
	\item провести анализ предметной области, сформулировать критерии сравнения методов обновления гипертекстовых документов.
	\item формализовать алгоритм Fiber обновления гипертекстового документа.
	\item сравнить и проанализировать трудоёмкости изученных алгоритмов;
	\item сделать выводы об эффективности использования алгоритма Fiber по сравнению с другими существующими алгоритмами.
\end{enumerate}


\pagebreak